% Clinical Study Censoring Visualization using our publication-grade styles
% This visualization shows patient follow-up in a longitudinal clinical study

\begin{figure}[htbp]
    \centering
    \begin{tikzpicture}
        \begin{axis}[
            publication,
            width=0.8\textwidth,
            height=0.4\textwidth,
            title={Patient Follow-up and Censoring in a Clinical Trial},
            xlabel={Study Time (months)},
            ylabel={Patients},
            ytick={1,2,3,4,5,6,7,8,9,10,11,12,13,14,15},
            yticklabels={P01, P02, P03, P04, P05, P06, P07, P08, P09, P10, P11, P12, P13, P14, P15},
            xmin=0,
            xmax=36,
            ymin=0.5,
            ymax=15.5,
            xmajorgrids=true,
            grid style={octonary!20, very thin},
            y dir=reverse,
            legend pos=north east,
            legend style={
                font=\small,
                cells={anchor=west}
            },
            clip=false
        ]
            % Study timeline markers
            \draw[thick, gray] (0,0.5) -- (0,15.5);
            \draw[thick, gray] (36,0.5) -- (36,15.5);
            \node[above, font=\small] at (0,0.5) {Study Start};
            \node[above, font=\small] at (36,0.5) {Study End};
            
            % Patient 1: Complete follow-up, event at the end
            \addplot[
                primaryDark,
                thick,
                mark=none
            ] coordinates {
                (0,1) (30,1)
            };
            
            % Event mark (e.g., progression)
            \node[circle, fill=event1Color, inner sep=2pt] at (30,1) {};
            
            % Patient 2: Complete follow-up without event
            \addplot[
                primaryDark,
                thick,
                mark=none
            ] coordinates {
                (0,2) (36,2)
            };
            
            % Add censoring mark at study end
            \draw[censoringColor, very thick] 
                (36-0.7,2-0.7) -- (36+0.7,2+0.7);
            \draw[censoringColor, very thick] 
                (36-0.7,2+0.7) -- (36+0.7,2-0.7);
                
            % Patient 3: Early enrollment, event
            \addplot[
                primaryDark,
                thick,
                mark=none
            ] coordinates {
                (4,3) (22,3)
            };
            
            % Event mark
            \node[circle, fill=event1Color, inner sep=2pt] at (22,3) {};
            
            % Patient 4: Late enrollment, no event
            \addplot[
                primaryDark,
                thick,
                mark=none
            ] coordinates {
                (12,4) (36,4)
            };
            
            % Add censoring mark at study end
            \draw[censoringColor, very thick] 
                (36-0.7,4-0.7) -- (36+0.7,4+0.7);
            \draw[censoringColor, very thick] 
                (36-0.7,4+0.7) -- (36+0.7,4-0.7);
                
            % Patient 5: Lost to follow-up
            \addplot[
                primaryDark,
                thick,
                mark=none
            ] coordinates {
                (0,5) (18,5)
            };
            
            % Add censoring mark
            \draw[censoringColor, very thick] 
                (18-0.7,5-0.7) -- (18+0.7,5+0.7);
            \draw[censoringColor, very thick] 
                (18-0.7,5+0.7) -- (18+0.7,5-0.7);
                
            % Patient 6: Dropped out
            \addplot[
                primaryDark,
                thick,
                mark=none
            ] coordinates {
                (0,6) (14,6)
            };
            
            % Add censoring mark
            \draw[censoringColor, very thick] 
                (14-0.7,6-0.7) -- (14+0.7,6+0.7);
            \draw[censoringColor, very thick] 
                (14-0.7,6+0.7) -- (14+0.7,6-0.7);
                
            % Patient 7: Competing event (e.g., different progression type)
            \addplot[
                primaryDark,
                thick,
                mark=none
            ] coordinates {
                (0,7) (26,7)
            };
            
            % Competing event mark
            \node[circle, fill=event2Color, inner sep=2pt] at (26,7) {};
            
            % Patient 8: Death (as a separate event type)
            \addplot[
                primaryDark,
                thick,
                mark=none
            ] coordinates {
                (0,8) (20,8)
            };
            
            % Death mark
            \node[circle, fill=quaternary, inner sep=2pt] at (20,8) {};
            
            % Patient 9: Late enrollment, competing event
            \addplot[
                primaryDark,
                thick,
                mark=none
            ] coordinates {
                (8,9) (24,9)
            };
            
            % Competing event mark
            \node[circle, fill=event2Color, inner sep=2pt] at (24,9) {};
            
            % Patient 10: Very late enrollment, ongoing at study end
            \addplot[
                primaryDark,
                thick,
                mark=none
            ] coordinates {
                (28,10) (36,10)
            };
            
            % Add censoring mark at study end
            \draw[censoringColor, very thick] 
                (36-0.7,10-0.7) -- (36+0.7,10+0.7);
            \draw[censoringColor, very thick] 
                (36-0.7,10+0.7) -- (36+0.7,10-0.7);
                
            % Patient 11: Event very early
            \addplot[
                primaryDark,
                thick,
                mark=none
            ] coordinates {
                (0,11) (6,11)
            };
            
            % Event mark
            \node[circle, fill=event1Color, inner sep=2pt] at (6,11) {};
            
            % Patient 12: Interval censoring - missed visits, then event found
            \addplot[
                primaryDark,
                thick,
                mark=none
            ] coordinates {
                (0,12) (10,12)
            };
            
            % Missed follow-up
            \addplot[
                primaryDark,
                thick,
                dashed
            ] coordinates {
                (10,12) (20,12)
            };
            
            \addplot[
                primaryDark,
                thick,
                mark=none
            ] coordinates {
                (20,12) (20,12)
            };
            
            % Event mark
            \node[circle, fill=event1Color, inner sep=2pt] at (20,12) {};
            
            % Patient 13: Death after progression
            \addplot[
                primaryDark,
                thick,
                mark=none
            ] coordinates {
                (0,13) (15,13)
            };
            
            % Event mark (progression)
            \node[circle, fill=event1Color, inner sep=2pt] at (15,13) {};
            
            % Continue observation after progression
            \addplot[
                primaryDark,
                thick,
                dotted
            ] coordinates {
                (15,13) (28,13)
            };
            
            % Death mark
            \node[circle, fill=quaternary, inner sep=2pt] at (28,13) {};
            
            % Patient 14: Multiple events
            \addplot[
                primaryDark,
                thick,
                mark=none
            ] coordinates {
                (0,14) (12,14)
            };
            
            % First event
            \node[circle, fill=event1Color, inner sep=2pt] at (12,14) {};
            
            % Continue to second event
            \addplot[
                primaryDark,
                thick, 
                dotted
            ] coordinates {
                (12,14) (26,14)
            };
            
            % Second event
            \node[circle, fill=event3Color, inner sep=2pt] at (26,14) {};
            
            % Patient 15: Treatment discontinuation 
            \addplot[
                primaryDark,
                thick,
                mark=none
            ] coordinates {
                (0,15) (16,15)
            };
            
            % Treatment discontinuation mark (triangle)
            \node[regular polygon, regular polygon sides=3, fill=octonaryDark, inner sep=1.5pt, rotate=90] at (16,15) {};
            
            % Different line style after treatment
            \addplot[
                primaryDark,
                thick,
                dashed
            ] coordinates {
                (16,15) (32,15)
            };
            
            % Add censoring mark
            \draw[censoringColor, very thick] 
                (32-0.7,15-0.7) -- (32+0.7,15+0.7);
            \draw[censoringColor, very thick] 
                (32-0.7,15+0.7) -- (32+0.7,15-0.7);
                
            % Add colored vertical assessment period markers - avoiding \foreach
            \draw[octonary!40, thin] (6,0.5) -- (6,15.5);
            \draw[octonary!40, thin] (12,0.5) -- (12,15.5);
            \draw[octonary!40, thin] (18,0.5) -- (18,15.5);
            \draw[octonary!40, thin] (24,0.5) -- (24,15.5);
            \draw[octonary!40, thin] (30,0.5) -- (30,15.5);
            
            \node[below, font=\scriptsize, text=octonary!80] at (6,15.5) {Assess.};
            \node[below, font=\scriptsize, text=octonary!80] at (12,15.5) {Assess.};
            \node[below, font=\scriptsize, text=octonary!80] at (18,15.5) {Assess.};
            \node[below, font=\scriptsize, text=octonary!80] at (24,15.5) {Assess.};
            \node[below, font=\scriptsize, text=octonary!80] at (30,15.5) {Assess.};
            
            % Add annotations
            \draw[<-, gray, thin] (31,1.3) -- (34,2);
            \node[anchor=west, gray, font=\scriptsize] at (34,2) {Progression};
            
            \draw[<-, gray, thin] (36.5,2.3) -- (38,3);
            \node[anchor=west, gray, font=\scriptsize] at (38,3) {Study completion};
            
            \draw[<-, gray, thin] (19,5.3) -- (22,6);
            \node[anchor=west, gray, font=\scriptsize] at (22,6) {Lost to follow-up};
            
            \draw[<-, gray, thin] (26.5,7.3) -- (29,8);
            \node[anchor=west, gray, font=\scriptsize] at (29,8) {Competing event};
            
            \draw[<-, gray, thin] (20.5,8.3) -- (23,9);
            \node[anchor=west, gray, font=\scriptsize] at (23,9) {Death};
            
            \draw[<-, gray, thin] (15,12.5) -- (16,11.7);
            \node[anchor=west, gray, font=\scriptsize] at (16,11.7) {Interval censoring};
            
            \draw[<-, gray, thin] (27.3,13.3) -- (30,14);
            \node[anchor=west, gray, font=\scriptsize] at (30,14) {Death after progression};
            
            \draw[<-, gray, thin] (17,15.3) -- (19,16);
            \node[anchor=west, gray, font=\scriptsize] at (19,16) {Treatment discontinuation};
            
            % Legend entries
            \addlegendimage{primaryDark, thick}
            \addlegendentry{Active follow-up}
            
            \addlegendimage{primaryDark, thick, dashed}
            \addlegendentry{Missed/unobserved period}
            
            \addlegendimage{primaryDark, thick, dotted}
            \addlegendentry{Post-progression follow-up}
            
            \addlegendimage{mark=*, mark size=3pt, mark options={fill=event1Color, solid}, only marks}
            \addlegendentry{Primary event}
            
            \addlegendimage{mark=*, mark size=3pt, mark options={fill=event2Color, solid}, only marks}
            \addlegendentry{Competing event}
            
            \addlegendimage{mark=*, mark size=3pt, mark options={fill=event3Color, solid}, only marks}
            \addlegendentry{Secondary event}
            
            \addlegendimage{mark=*, mark size=3pt, mark options={fill=quaternary, solid}, only marks}
            \addlegendentry{Death}
            
            \addlegendimage{censoringColor, thick, mark=+, mark size=3pt}
            \addlegendentry{Censored}
            
            \addlegendimage{regular polygon, regular polygon sides=3, fill=octonaryDark, inner sep=1.5pt}
            \addlegendentry{Treatment discontinuation}
        \end{axis}
    \end{tikzpicture}
    \caption{Visualization of patient follow-up in a 36-month clinical trial. The plot shows various scenarios including complete follow-up, early/late enrollment, loss to follow-up, treatment discontinuation, competing events, and interval censoring. Vertical lines represent scheduled assessment visits at 6-month intervals. This visual representation illustrates the complexity of longitudinal data in clinical studies and the various types of censoring that must be accounted for in survival analysis.}
    \label{fig:clinical-censoring}
\end{figure}

% Time-to-event analysis visualization
\begin{figure}[htbp]
    \centering
    \begin{tikzpicture}
        \begin{axis}[
            publication,
            width=0.8\textwidth,
            height=0.4\textwidth,
            title={Time-to-Event Analysis in a Clinical Trial},
            xlabel={Months from Randomization},
            ylabel={Event-free Proportion},
            xmin=0,
            xmax=36,
            ymin=0,
            ymax=1.05,
            ymajorgrids=true,
            grid style={octonary!20, very thin},
            legend pos=south west,
            legend style={
                font=\small,
                cells={anchor=west}
            }
        ]
            % Kaplan-Meier Curve for Treatment Group
            \addplot[
                survivalFunctionColor,
                thick,
                const plot mark right,
                mark=none
            ] coordinates {
                (0, 1.0) (3, 1.0) (6, 0.95) (9, 0.95) (12, 0.90) 
                (15, 0.88) (18, 0.85) (21, 0.82) (24, 0.82) 
                (27, 0.79) (30, 0.76) (33, 0.76) (36, 0.75)
            };
            
            % Add confidence bands for treatment group
            \addplot[
                survivalFunctionColor!20,
                opacity=0.5,
                const plot mark right,
                mark=none,
                name path=upperT
            ] coordinates {
                (0, 1.0) (3, 1.0) (6, 0.98) (9, 0.98) (12, 0.94) 
                (15, 0.92) (18, 0.89) (21, 0.87) (24, 0.87) 
                (27, 0.84) (30, 0.82) (33, 0.82) (36, 0.81)
            };
            
            \addplot[
                survivalFunctionColor!20,
                opacity=0.5,
                const plot mark right,
                mark=none,
                name path=lowerT
            ] coordinates {
                (0, 1.0) (3, 1.0) (6, 0.92) (9, 0.92) (12, 0.86) 
                (15, 0.84) (18, 0.81) (21, 0.77) (24, 0.77) 
                (27, 0.74) (30, 0.70) (33, 0.70) (36, 0.69)
            };
            
            % Fill between confidence bands
            \addplot[
                survivalFunctionColor!20,
                fill opacity=0.3
            ] fill between[of=upperT and lowerT];
            
            % Kaplan-Meier Curve for Control Group
            \addplot[
                quaternary,
                thick,
                const plot mark right,
                mark=none
            ] coordinates {
                (0, 1.0) (3, 0.98) (6, 0.90) (9, 0.85) (12, 0.78) 
                (15, 0.72) (18, 0.65) (21, 0.60) (24, 0.58) 
                (27, 0.55) (30, 0.52) (33, 0.52) (36, 0.50)
            };
            
            % Add confidence bands for control group
            \addplot[
                quaternary!20,
                opacity=0.5,
                const plot mark right,
                mark=none,
                name path=upperC
            ] coordinates {
                (0, 1.0) (3, 1.0) (6, 0.95) (9, 0.90) (12, 0.84) 
                (15, 0.78) (18, 0.71) (21, 0.66) (24, 0.64) 
                (27, 0.61) (30, 0.58) (33, 0.58) (36, 0.56)
            };
            
            \addplot[
                quaternary!20,
                opacity=0.5,
                const plot mark right,
                mark=none,
                name path=lowerC
            ] coordinates {
                (0, 1.0) (3, 0.96) (6, 0.85) (9, 0.80) (12, 0.72) 
                (15, 0.66) (18, 0.59) (21, 0.54) (24, 0.52) 
                (27, 0.49) (30, 0.46) (33, 0.46) (36, 0.44)
            };
            
            % Fill between confidence bands
            \addplot[
                quaternary!20,
                fill opacity=0.3
            ] fill between[of=upperC and lowerC];
            
            % Add censoring marks
            % Treatment group
            \addplot[
                only marks,
                mark=+,
                mark size=4pt,
                mark options={solid, survivalFunctionColor}
            ] coordinates {
                (6, 0.95) (15, 0.88) (18, 0.85) (27, 0.79) (36, 0.75)
            };
            
            % Control group
            \addplot[
                only marks,
                mark=+,
                mark size=4pt,
                mark options={solid, quaternary}
            ] coordinates {
                (3, 0.98) (9, 0.85) (15, 0.72) (24, 0.58) (36, 0.50)
            };
            
            % Add simplified risk table
            \node[anchor=north west, align=left, font=\footnotesize] at (0,-0.12) {
                \begin{tabular}{@{}l@{\hspace{1pt}}|@{\hspace{1pt}}c@{\hspace{1pt}}c@{\hspace{1pt}}c@{\hspace{1pt}}c@{\hspace{1pt}}c@{\hspace{1pt}}c@{\hspace{1pt}}c@{}}
                \textbf{At risk} & \textbf{0} & \textbf{6} & \textbf{12} & \textbf{18} & \textbf{24} & \textbf{30} & \textbf{36} \\
                \hline
                Treat & 100 & 95 & 90 & 85 & 82 & 76 & 75 \\
                Ctrl & 100 & 90 & 78 & 65 & 58 & 52 & 50 \\
                \end{tabular}
            };
            
            % Add p-value and hazard ratio (simplified)
            \node[anchor=south east, align=right, font=\footnotesize] at (36,1.02) {
                \begin{tabular}{r}
                HR: 0.57 (0.41-0.78) \\
                $p = 0.0015$
                \end{tabular}
            };
            
            % Label median survival times
            \draw[dashed, quaternary] (0,0.5) -- (36,0.5);
            \draw[dashed, quaternary] (18,0.5) -- (18,0);
            \node[anchor=north west, font=\footnotesize, text=quaternary] at (18,0.48) {Median: 18m};
            
            \draw[dashed, survivalFunctionColor] (0,0.5) -- (36,0.5);
            \draw[dashed, survivalFunctionColor] (36,0.5) -- (36,0);
            \node[anchor=south east, font=\footnotesize, text=survivalFunctionColor] at (36,0.52) {Median: >36m};
            
            % Legend
            \addlegendentry{Treatment arm};
            \addlegendentry{95\% CI (Treatment)};
            \addlegendentry{Control arm};
            \addlegendentry{95\% CI (Control)};
            \addlegendentry{Censored (Treatment)};
            \addlegendentry{Censored (Control)};
        \end{axis}
    \end{tikzpicture}
    \caption{Kaplan-Meier event-free survival curves comparing treatment and control arms in a clinical trial with 36-month follow-up. Shaded areas represent 95\% confidence intervals, and '+' marks indicate censored observations. The treatment arm shows a significant improvement in event-free survival with a hazard ratio of 0.57 (95\% CI: 0.41-0.78) and a log-rank p-value of 0.0015. The median event-free survival is 18 months in the control arm and has not been reached in the treatment arm at the 36-month study endpoint. The number of patients at risk is shown below the graph.}
    \label{fig:km-clinical}
\end{figure}