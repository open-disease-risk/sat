% Time-to-event data visualization with five subjects
% Following the publication-grade style guide

\begin{figure}[htbp]
    \centering
    \begin{tikzpicture}
        \begin{axis}[
            publication,
            width=0.8\textwidth,
            height=0.4\textwidth,
            title={Time-to-Event Data for Five Subjects},
            xlabel={Time (months)},
            ylabel={Subject},
            ytick={1,2,3,4,5},
            yticklabels={Subject 1, Subject 2, Subject 3, Subject 4, Subject 5},
            xmin=0,
            xmax=60,
            ymin=0.25,
            ymax=5.75,
            xmajorgrids=true,
            grid style={octonary!20, very thin},
            y dir=reverse,
            legend pos=north east,
            legend style={
                font=\small,
                cells={anchor=west}
            }
        ]
            % Study timeline reference points
            \draw[thin, octonary!60, dashed] (0,0.25) -- (0,5.75);
            \draw[thin, octonary!60, dashed] (24,0.25) -- (24,5.75);
            \draw[thin, octonary!60, dashed] (48,0.25) -- (48,5.75);
            \node[below, font=\footnotesize, text=octonary] at (0,5.75) {Study start};
            \node[below, font=\footnotesize, text=octonary] at (24,5.75) {2 years};
            \node[below, font=\footnotesize, text=octonary] at (48,5.75) {4 years};
            
            % Subject 1: Standard right censoring
            \addplot[
                primaryDark,
                thick,
                mark=none
            ] coordinates {
                (0,1) (30,1)
            };
            
            % Add censoring mark
            \draw[censoringColor, very thick] 
                (30-0.7,1-0.7) -- (30+0.7,1+0.7);
            \draw[censoringColor, very thick] 
                (30-0.7,1+0.7) -- (30+0.7,1-0.7);
                
            % Subject 2: Event observed
            \addplot[
                primaryDark,
                thick,
                mark=none
            ] coordinates {
                (0,2) (40,2)
            };
            
            % Event mark
            \node[circle, fill=event1Color, inner sep=2pt] at (40,2) {};
            
            % Subject 3: Left truncation (delayed entry) + event
            \addplot[
                primaryDark,
                thick,
                mark=none,
                opacity=0.3
            ] coordinates {
                (0,3) (12,3)
            };
            
            \addplot[
                primaryDark,
                thick,
                mark=none
            ] coordinates {
                (12,3) (36,3)
            };
            
            % Event mark
            \node[circle, fill=event1Color, inner sep=2pt] at (36,3) {};
            
            % Subject 4: Left truncation + right censoring
            \addplot[
                primaryDark,
                thick,
                mark=none,
                opacity=0.3
            ] coordinates {
                (0,4) (18,4)
            };
            
            \addplot[
                primaryDark,
                thick,
                mark=none
            ] coordinates {
                (18,4) (50,4)
            };
            
            % Add censoring mark
            \draw[censoringColor, very thick] 
                (50-0.7,4-0.7) -- (50+0.7,4+0.7);
            \draw[censoringColor, very thick] 
                (50-0.7,4+0.7) -- (50+0.7,4-0.7);
                
            % Subject 5: Competing risk
            \addplot[
                primaryDark,
                thick,
                mark=none
            ] coordinates {
                (0,5) (44,5)
            };
            
            % Competing risk event mark
            \node[circle, fill=event2Color, inner sep=2pt] at (44,5) {};
            
            % Labels with arrows - positioned to avoid crossing lines
            % Subject 1: Right censoring
            \draw[->, thin, gray] (30,0.6) -- (30,0.9);
            \node[below, align=center, font=\footnotesize, text=gray] at (30,0.5) {Right censoring\\(lost to follow-up)};
            
            % Subject 2: Event
            \draw[->, thin, gray] (40,1.6) -- (40,1.9);
            \node[below, align=center, font=\footnotesize, text=gray] at (40,1.5) {Event observed};
            
            % Subject 3: Left truncation
            \draw[->, thin, gray] (8,2.6) -- (8,2.9);
            \node[below, align=center, font=\footnotesize, text=gray] at (8,2.5) {Left truncation\\(delayed entry)};
            
            % Subject 3: Event
            \draw[->, thin, gray] (36,2.6) -- (36,2.9);
            \node[below, align=center, font=\footnotesize, text=gray] at (36,2.5) {Event};
            
            % Subject 4: Left truncation
            \draw[->, thin, gray] (12,3.6) -- (12,3.9);
            \node[below, align=center, font=\footnotesize, text=gray] at (12,3.5) {Left truncation};
            
            % Subject 4: Right censoring
            \draw[->, thin, gray] (50,3.6) -- (50,3.9);
            \node[below, align=center, font=\footnotesize, text=gray] at (50,3.5) {Right censoring\\(end of study)};
            
            % Subject 5: Competing risk
            \draw[->, thin, gray] (44,4.6) -- (44,4.9);
            \node[below, align=center, font=\footnotesize, text=gray] at (44,4.5) {Competing risk\\event};
            
            % Add a legend at the bottom of the plot instead of top-right
            \legend{}
            
            % Create a separate legend below the plot using nodes
            \coordinate (legendpos) at (rel axis cs:0.5,-0.25);
            
            % Symbol for observation time
            \node[inner sep=0] at ($(legendpos)+(-4.5,0)$) {\tikz{\draw[primaryDark, thick] (-0.15,0) -- (0.15,0);}};
            \node[right, font=\footnotesize] at ($(legendpos)+(-4.3,0)$) {Observation time};
            
            % Symbol for unobserved time
            \node[inner sep=0] at ($(legendpos)+(-1.5,0)$) {\tikz{\draw[primaryDark, thick, opacity=0.3] (-0.15,0) -- (0.15,0);}};
            \node[right, font=\footnotesize] at ($(legendpos)+(-1.3,0)$) {Unobserved time};
            
            % Symbol for event of interest
            \node[inner sep=0] at ($(legendpos)+(1.5,0)$) {\tikz{\draw[mark size=3pt, mark=*, mark options={fill=event1Color}] plot coordinates {(0,0)};}};
            \node[right, font=\footnotesize] at ($(legendpos)+(1.7,0)$) {Event of interest};
            
            % Symbol for competing event
            \node[inner sep=0] at ($(legendpos)+(-4.5,-0.5)$) {\tikz{\draw[mark size=3pt, mark=*, mark options={fill=event2Color}] plot coordinates {(0,0)};}};
            \node[right, font=\footnotesize] at ($(legendpos)+(-4.3,-0.5)$) {Competing event};
            
            % Symbol for censoring
            \node[inner sep=0] at ($(legendpos)+(-1.5,-0.5)$) {\tikz{\draw[censoringColor, very thick] (-0.1,-0.1) -- (0.1,0.1); \draw[censoringColor, very thick] (-0.1,0.1) -- (0.1,-0.1);}};
            \node[right, font=\footnotesize] at ($(legendpos)+(-1.3,-0.5)$) {Censoring};
        \end{axis}
    \end{tikzpicture}
    \caption{Time-to-event data for five subjects showing various censoring and event patterns. Subject 1 experiences right censoring due to loss to follow-up. Subject 2 has an observed event. Subject 3 enters the study late (left truncation) and experiences an event. Subject 4 enters late and is right-censored at the end of the study. Subject 5 experiences a competing risk event.}
    \label{fig:time-to-event-subjects}
\end{figure}

% Visualization of left censoring examples
\begin{figure}[htbp]
    \centering
    \begin{tikzpicture}
        \begin{axis}[
            publication,
            width=0.8\textwidth,
            height=0.4\textwidth,
            title={Examples of Left Censoring in Survival Analysis},
            xlabel={Time (months)},
            ylabel={Case},
            ytick={1,2,3,4},
            yticklabels={Case 1, Case 2, Case 3, Case 4},
            xmin=0,
            xmax=60,
            ymin=0.25,
            ymax=4.75,
            xmajorgrids=true,
            grid style={octonary!20, very thin},
            y dir=reverse,
            legend pos=north east,
            legend style={
                font=\small,
                cells={anchor=west}
            }
        ]
            % Timeline reference points
            \draw[thin, octonary!60, dashed] (10,0.25) -- (10,4.75);
            \node[below, font=\footnotesize, text=octonary] at (10,4.75) {Screening time};
            
            % Case 1: Event known to occur before screening
            % Unobserved event time
            \addplot[
                primaryDark,
                thick,
                mark=none,
                opacity=0.3
            ] coordinates {
                (0,1) (7,1)
            };
            
            % Event mark (not directly observed)
            \node[circle, fill=event1Color, inner sep=2pt, opacity=0.5] at (7,1) {};
            
            % Observed after event
            \addplot[
                primaryDark,
                thick,
                mark=none
            ] coordinates {
                (10,1) (30,1)
            };
            
            % Case 2: Condition present at screening
            % Unobserved event time
            \addplot[
                primaryDark,
                thick,
                mark=none,
                opacity=0.3
            ] coordinates {
                (0,2) (10,2)
            };
            
            % Event before screening
            \node[circle, fill=event1Color, inner sep=2pt, opacity=0.5] at (5,2) {};
            
            % Left censoring mark at screening
            \draw[censoringColor, very thick, rotate=180] 
                (10-0.7,2-0.7) -- (10+0.7,2+0.7);
            \draw[censoringColor, very thick, rotate=180] 
                (10-0.7,2+0.7) -- (10+0.7,2-0.7);
                
            % Observed after positive screening
            \addplot[
                primaryDark,
                thick,
                mark=none
            ] coordinates {
                (10,2) (40,2)
            };
            
            % Case 3: Test detects condition with delay, then event
            % Unobserved period
            \addplot[
                primaryDark,
                thick,
                mark=none,
                opacity=0.3
            ] coordinates {
                (0,3) (10,3)
            };
            
            % Tested negative at first
            \addplot[
                primaryDark,
                thick,
                mark=none
            ] coordinates {
                (10,3) (25,3)
            };
            
            % Test detects condition
            \node[diamond, fill=event3Color, inner sep=1.5pt] at (25,3) {};
            
            % Followed until event
            \addplot[
                primaryDark,
                thick,
                mark=none
            ] coordinates {
                (25,3) (45,3)
            };
            
            % Event mark
            \node[circle, fill=event1Color, inner sep=2pt] at (45,3) {};
            
            % Case 4: Left and right censoring
            % Left censored unobserved period
            \addplot[
                primaryDark,
                thick,
                mark=none,
                opacity=0.3
            ] coordinates {
                (0,4) (10,4)
            };
            
            % Left censoring mark
            \draw[censoringColor, very thick, rotate=180] 
                (10-0.7,4-0.7) -- (10+0.7,4+0.7);
            \draw[censoringColor, very thick, rotate=180] 
                (10-0.7,4+0.7) -- (10+0.7,4-0.7);
                
            % Observed period
            \addplot[
                primaryDark,
                thick,
                mark=none
            ] coordinates {
                (10,4) (35,4)
            };
            
            % Right censoring mark
            \draw[censoringColor, very thick] 
                (35-0.7,4-0.7) -- (35+0.7,4+0.7);
            \draw[censoringColor, very thick] 
                (35-0.7,4+0.7) -- (35+0.7,4-0.7);
                
            % Labels with arrows - positioned to avoid crossing lines
            % Case 1
            \draw[->, thin, gray] (7,0.6) -- (7,0.9);
            \node[below, align=center, font=\footnotesize, text=gray] at (7,0.5) {Event occurred\\before screening};
            
            \draw[->, thin, gray] (20,0.6) -- (20,0.9);
            \node[below, align=center, font=\footnotesize, text=gray] at (20,0.5) {Observed with\\condition};
            
            % Case 2
            \draw[->, thin, gray] (5,1.6) -- (5,1.9);
            \node[below, align=center, font=\footnotesize, text=gray] at (5,1.5) {Unknown\\event time};
            
            \draw[->, thin, gray] (10,1.6) -- (10,1.9);
            \node[below, align=center, font=\footnotesize, text=gray] at (10,1.5) {Left censoring\\at screening};
            
            % Case 3
            \draw[->, thin, gray] (18,2.6) -- (18,2.9);
            \node[below, align=center, font=\footnotesize, text=gray] at (18,2.5) {Test negative};
            
            \draw[->, thin, gray] (25,2.6) -- (25,2.9);
            \node[below, align=center, font=\footnotesize, text=gray] at (25,2.5) {Test positive};
            
            \draw[->, thin, gray] (45,2.6) -- (45,2.9);
            \node[below, align=center, font=\footnotesize, text=gray] at (45,2.5) {Event};
            
            % Case 4
            \draw[->, thin, gray] (10,3.6) -- (10,3.9);
            \node[below, align=center, font=\footnotesize, text=gray] at (10,3.5) {Left censoring};
            
            \draw[->, thin, gray] (35,3.6) -- (35,3.9);
            \node[below, align=center, font=\footnotesize, text=gray] at (35,3.5) {Right censoring};
            
            % Add a legend at the bottom of the plot instead of top-right
            \legend{}
            
            % Create a separate legend below the plot using nodes
            \coordinate (legendpos) at (rel axis cs:0.5,-0.3);
            
            % Row 1
            % Symbol for observed follow-up
            \node[inner sep=0] at ($(legendpos)+(-4.5,0.3)$) {\tikz{\draw[primaryDark, thick] (-0.15,0) -- (0.15,0);}};
            \node[right, font=\footnotesize] at ($(legendpos)+(-4.3,0.3)$) {Observed follow-up};
            
            % Symbol for unobserved time
            \node[inner sep=0] at ($(legendpos)+(-1.5,0.3)$) {\tikz{\draw[primaryDark, thick, opacity=0.3] (-0.15,0) -- (0.15,0);}};
            \node[right, font=\footnotesize] at ($(legendpos)+(-1.3,0.3)$) {Unobserved time};
            
            % Symbol for observed event
            \node[inner sep=0] at ($(legendpos)+(1.5,0.3)$) {\tikz{\draw[mark size=3pt, mark=*, mark options={fill=event1Color}] plot coordinates {(0,0)};}};
            \node[right, font=\footnotesize] at ($(legendpos)+(1.7,0.3)$) {Observed event};
            
            % Row 2
            % Symbol for unobserved event
            \node[inner sep=0] at ($(legendpos)+(-4.5,0)$) {\tikz{\draw[mark size=3pt, mark=*, mark options={fill=event1Color, opacity=0.5}] plot coordinates {(0,0)};}};
            \node[right, font=\footnotesize] at ($(legendpos)+(-4.3,0)$) {Unobserved event};
            
            % Symbol for test detection
            \node[inner sep=0] at ($(legendpos)+(-1.5,0)$) {\tikz{\draw[mark size=3pt, mark=diamond, mark options={fill=event3Color}] plot coordinates {(0,0)};}};
            \node[right, font=\footnotesize] at ($(legendpos)+(-1.3,0)$) {Test detection};
            
            % Row 3
            % Symbol for right censoring
            \node[inner sep=0] at ($(legendpos)+(-4.5,-0.3)$) {\tikz{\draw[censoringColor, very thick] (-0.1,-0.1) -- (0.1,0.1); \draw[censoringColor, very thick] (-0.1,0.1) -- (0.1,-0.1);}};
            \node[right, font=\footnotesize] at ($(legendpos)+(-4.3,-0.3)$) {Right censoring};
            
            % Symbol for left censoring
            \node[inner sep=0] at ($(legendpos)+(-1.5,-0.3)$) {\tikz{\draw[censoringColor, very thick, rotate=180] (-0.1,-0.1) -- (0.1,0.1); \draw[censoringColor, very thick, rotate=180] (-0.1,0.1) -- (0.1,-0.1);}};
            \node[right, font=\footnotesize] at ($(legendpos)+(-1.3,-0.3)$) {Left censoring};
        \end{axis}
    \end{tikzpicture}
    \caption{Examples of left censoring in survival analysis. Case 1 shows an event known to have occurred before screening, with the exact time unknown. Case 2 shows a condition detected at the initial screening (left censored). Case 3 shows delayed detection where a test initially misses the condition but later detects it. Case 4 shows both left and right censoring in the same subject. For left censoring, we only know that the event occurred before a certain time point, in contrast to right censoring where we know it occurred after.}
    \label{fig:left-censoring-examples}
\end{figure}