% ===================================
% COLOR SCHEME DEFINITIONS
% ===================================
% This file defines a consistent color palette for the entire book
% To use, include this file in book.tex with % ===================================
% COLOR SCHEME DEFINITIONS
% ===================================
% This file defines a consistent color palette for the entire book
% To use, include this file in book.tex with % ===================================
% COLOR SCHEME DEFINITIONS
% ===================================
% This file defines a consistent color palette for the entire book
% To use, include this file in book.tex with % ===================================
% COLOR SCHEME DEFINITIONS
% ===================================
% This file defines a consistent color palette for the entire book
% To use, include this file in book.tex with \input{book-sections/color-scheme.tex}

% Define primary colors
\definecolor{primary}{RGB}{31, 119, 180}       % Blue
\definecolor{secondary}{RGB}{255, 127, 14}     % Orange
\definecolor{tertiary}{RGB}{44, 160, 44}       % Green
\definecolor{quaternary}{RGB}{214, 39, 40}     % Red
\definecolor{quinary}{RGB}{148, 103, 189}      % Purple
\definecolor{senary}{RGB}{140, 86, 75}         % Brown
\definecolor{septenary}{RGB}{227, 119, 194}    % Pink
\definecolor{octonary}{RGB}{127, 127, 127}     % Gray
\definecolor{nonary}{RGB}{188, 189, 34}        % Olive
\definecolor{denary}{RGB}{23, 190, 207}        % Cyan

% Define light versions for fills and backgrounds
\definecolor{primaryLight}{RGB}{174, 199, 232}
\definecolor{secondaryLight}{RGB}{255, 187, 120}
\definecolor{tertiaryLight}{RGB}{152, 223, 138}
\definecolor{quaternaryLight}{RGB}{255, 152, 150}
\definecolor{quinaryLight}{RGB}{197, 176, 213}
\definecolor{senaryLight}{RGB}{196, 156, 148}
\definecolor{septenaryLight}{RGB}{247, 182, 210}
\definecolor{octonaryLight}{RGB}{199, 199, 199}
\definecolor{nonaryLight}{RGB}{219, 219, 141}
\definecolor{denaryLight}{RGB}{158, 218, 229}

% Define dark versions for text and lines
\definecolor{primaryDark}{RGB}{17, 63, 96}
\definecolor{secondaryDark}{RGB}{179, 89, 10}
\definecolor{tertiaryDark}{RGB}{22, 80, 22}
\definecolor{quaternaryDark}{RGB}{136, 22, 22}
\definecolor{quinaryDark}{RGB}{94, 66, 121}
\definecolor{senaryDark}{RGB}{85, 52, 45}
\definecolor{septenaryDark}{RGB}{143, 75, 122}
\definecolor{octonaryDark}{RGB}{64, 64, 64}
\definecolor{nonaryDark}{RGB}{104, 104, 18}
\definecolor{denaryDark}{RGB}{14, 115, 124}

% ===================================
% TCOLORBOX STYLES
% ===================================

% Reset tcolorbox definitions with the new colors
\newtcolorbox{definitionbox}[1][]{
    enhanced,
    colback=primaryLight!60!white,
    colframe=primary!75!black,
    fonttitle=\bfseries,
    coltitle=white,
    attach boxed title to top left={yshift=-2mm, xshift=5mm},
    boxed title style={colback=primary!75!black},
    title=Definition,
    #1
}

\newtcolorbox{equationbox}[1][]{
    enhanced,
    colback=secondaryLight!60!white,
    colframe=secondary!75!black,
    fonttitle=\bfseries,
    breakable,
    coltitle=white,
    attach boxed title to top left={yshift=-2mm, xshift=5mm},
    boxed title style={colback=secondary!75!black},
    title=Equation,
    #1
}

\newtcolorbox{examplebox}[1][]{
    enhanced,
    colback=tertiaryLight!60!white,
    colframe=tertiary!75!black,
    fonttitle=\bfseries,
    coltitle=white,
    attach boxed title to top left={yshift=-2mm, xshift=5mm},
    boxed title style={colback=tertiary!75!black},
    title=Example,
    #1
}

\newtcolorbox{notebox}[1][]{
    enhanced,
    colback=quinaryLight!60!white,
    colframe=quinary!75!black,
    fonttitle=\bfseries,
    coltitle=white,
    attach boxed title to top left={yshift=-2mm, xshift=5mm},
    boxed title style={colback=quinary!75!black},
    title=Note,
    #1
}

% ===================================
% TIKZ AND PGFPLOTS STYLES
% ===================================

% TikZ styles
\tikzset{
    concept/.style={
        draw=primary,
        fill=primaryLight,
        rounded corners,
        thick,
        align=center
    },
    process/.style={
        draw=secondary,
        fill=secondaryLight,
        rectangle,
        thick,
        align=center
    },
    decision/.style={
        draw=tertiary,
        fill=tertiaryLight,
        diamond,
        thick,
        align=center
    },
    data/.style={
        draw=quaternary,
        fill=quaternaryLight,
        cylinder,
        thick,
        align=center
    },
    note/.style={
        draw=quinary,
        fill=quinaryLight,
        cloud,
        cloud puffs=15,
        thick,
        align=center
    },
    arrow/.style={
        ->,
        >=stealth,
        thick
    }
}

% PGFPlots styles
\pgfplotscreateplotcyclelist{conceptcolors}{
    {primary, mark=*, thick, mark options={fill=primary}},
    {secondary, mark=square*, thick, mark options={fill=secondary}},
    {tertiary, mark=triangle*, thick, mark options={fill=tertiary}},
    {quaternary, mark=diamond*, thick, mark options={fill=quaternary}},
    {quinary, mark=pentagon*, thick, mark options={fill=quinary}},
    {senary, mark=*, thick, mark options={fill=senary}},
    {septenary, mark=square*, thick, mark options={fill=septenary}},
    {octonary, mark=triangle*, thick, mark options={fill=octonary}},
    {nonary, mark=diamond*, thick, mark options={fill=nonary}},
    {denary, mark=pentagon*, thick, mark options={fill=denary}}
}

% ===================================
% CONCEPT-SPECIFIC COLORS
% ===================================

% Assign consistent colors to specific concepts
% These can be used across diagrams to ensure the same concept has the same color

% Survival functions and models
\colorlet{survivalFunctionColor}{primary}
\colorlet{hazardFunctionColor}{secondary}
\colorlet{censoringColor}{quaternary}
\colorlet{kaplameierColor}{tertiary}

% Model types
\colorlet{dsmColor}{quinary}
\colorlet{mensaColor}{septenary}
\colorlet{deephitColor}{senary}

% Loss functions
\colorlet{likelihoodLossColor}{primary}
\colorlet{rankingLossColor}{secondary}
\colorlet{regressionLossColor}{tertiary}
\colorlet{classificationLossColor}{quaternary}
\colorlet{auxiliaryLossColor}{quinary}

% Events
\colorlet{event1Color}{primary}
\colorlet{event2Color}{secondary}
\colorlet{event3Color}{tertiary}
\colorlet{event4Color}{quaternary}

% Risk groups
\colorlet{lowRiskColor}{tertiary}
\colorlet{mediumRiskColor}{primary}
\colorlet{highRiskColor}{quaternary}

% ===================================
% USAGE EXAMPLES
% ===================================
%
% For tcolorbox:
% \begin{definitionbox}[title=Custom Title]
% Content
% \end{definitionbox}
%
% For TikZ:
% \begin{tikzpicture}
%   \node[concept] (a) {Concept};
%   \node[process] (b) at (2,0) {Process};
%   \draw[arrow] (a) -- (b);
% \end{tikzpicture}
%
% For PGFPlots:
% \begin{tikzpicture}
%   \begin{axis}[cycle list name=conceptcolors]
%     \addplot coordinates {(0,0) (1,1) (2,4)};
%     \addplot coordinates {(0,0) (1,2) (2,3)};
%   \end{axis}
% \end{tikzpicture}
%
% For concept-specific colors:
% \begin{tikzpicture}
%   \draw[survivalFunctionColor, thick] plot coordinates {(0,1) (1,0.8) (2,0.6)};
%   \draw[hazardFunctionColor, thick] plot coordinates {(0,0.2) (1,0.4) (2,0.6)};
% \end{tikzpicture}


% Define primary colors
\definecolor{primary}{RGB}{31, 119, 180}       % Blue
\definecolor{secondary}{RGB}{255, 127, 14}     % Orange
\definecolor{tertiary}{RGB}{44, 160, 44}       % Green
\definecolor{quaternary}{RGB}{214, 39, 40}     % Red
\definecolor{quinary}{RGB}{148, 103, 189}      % Purple
\definecolor{senary}{RGB}{140, 86, 75}         % Brown
\definecolor{septenary}{RGB}{227, 119, 194}    % Pink
\definecolor{octonary}{RGB}{127, 127, 127}     % Gray
\definecolor{nonary}{RGB}{188, 189, 34}        % Olive
\definecolor{denary}{RGB}{23, 190, 207}        % Cyan

% Define light versions for fills and backgrounds
\definecolor{primaryLight}{RGB}{174, 199, 232}
\definecolor{secondaryLight}{RGB}{255, 187, 120}
\definecolor{tertiaryLight}{RGB}{152, 223, 138}
\definecolor{quaternaryLight}{RGB}{255, 152, 150}
\definecolor{quinaryLight}{RGB}{197, 176, 213}
\definecolor{senaryLight}{RGB}{196, 156, 148}
\definecolor{septenaryLight}{RGB}{247, 182, 210}
\definecolor{octonaryLight}{RGB}{199, 199, 199}
\definecolor{nonaryLight}{RGB}{219, 219, 141}
\definecolor{denaryLight}{RGB}{158, 218, 229}

% Define dark versions for text and lines
\definecolor{primaryDark}{RGB}{17, 63, 96}
\definecolor{secondaryDark}{RGB}{179, 89, 10}
\definecolor{tertiaryDark}{RGB}{22, 80, 22}
\definecolor{quaternaryDark}{RGB}{136, 22, 22}
\definecolor{quinaryDark}{RGB}{94, 66, 121}
\definecolor{senaryDark}{RGB}{85, 52, 45}
\definecolor{septenaryDark}{RGB}{143, 75, 122}
\definecolor{octonaryDark}{RGB}{64, 64, 64}
\definecolor{nonaryDark}{RGB}{104, 104, 18}
\definecolor{denaryDark}{RGB}{14, 115, 124}

% ===================================
% TCOLORBOX STYLES
% ===================================

% Reset tcolorbox definitions with the new colors
\newtcolorbox{definitionbox}[1][]{
    enhanced,
    colback=primaryLight!60!white,
    colframe=primary!75!black,
    fonttitle=\bfseries,
    coltitle=white,
    attach boxed title to top left={yshift=-2mm, xshift=5mm},
    boxed title style={colback=primary!75!black},
    title=Definition,
    #1
}

\newtcolorbox{equationbox}[1][]{
    enhanced,
    colback=secondaryLight!60!white,
    colframe=secondary!75!black,
    fonttitle=\bfseries,
    breakable,
    coltitle=white,
    attach boxed title to top left={yshift=-2mm, xshift=5mm},
    boxed title style={colback=secondary!75!black},
    title=Equation,
    #1
}

\newtcolorbox{examplebox}[1][]{
    enhanced,
    colback=tertiaryLight!60!white,
    colframe=tertiary!75!black,
    fonttitle=\bfseries,
    coltitle=white,
    attach boxed title to top left={yshift=-2mm, xshift=5mm},
    boxed title style={colback=tertiary!75!black},
    title=Example,
    #1
}

\newtcolorbox{notebox}[1][]{
    enhanced,
    colback=quinaryLight!60!white,
    colframe=quinary!75!black,
    fonttitle=\bfseries,
    coltitle=white,
    attach boxed title to top left={yshift=-2mm, xshift=5mm},
    boxed title style={colback=quinary!75!black},
    title=Note,
    #1
}

% ===================================
% TIKZ AND PGFPLOTS STYLES
% ===================================

% TikZ styles
\tikzset{
    concept/.style={
        draw=primary,
        fill=primaryLight,
        rounded corners,
        thick,
        align=center
    },
    process/.style={
        draw=secondary,
        fill=secondaryLight,
        rectangle,
        thick,
        align=center
    },
    decision/.style={
        draw=tertiary,
        fill=tertiaryLight,
        diamond,
        thick,
        align=center
    },
    data/.style={
        draw=quaternary,
        fill=quaternaryLight,
        cylinder,
        thick,
        align=center
    },
    note/.style={
        draw=quinary,
        fill=quinaryLight,
        cloud,
        cloud puffs=15,
        thick,
        align=center
    },
    arrow/.style={
        ->,
        >=stealth,
        thick
    }
}

% PGFPlots styles
\pgfplotscreateplotcyclelist{conceptcolors}{
    {primary, mark=*, thick, mark options={fill=primary}},
    {secondary, mark=square*, thick, mark options={fill=secondary}},
    {tertiary, mark=triangle*, thick, mark options={fill=tertiary}},
    {quaternary, mark=diamond*, thick, mark options={fill=quaternary}},
    {quinary, mark=pentagon*, thick, mark options={fill=quinary}},
    {senary, mark=*, thick, mark options={fill=senary}},
    {septenary, mark=square*, thick, mark options={fill=septenary}},
    {octonary, mark=triangle*, thick, mark options={fill=octonary}},
    {nonary, mark=diamond*, thick, mark options={fill=nonary}},
    {denary, mark=pentagon*, thick, mark options={fill=denary}}
}

% ===================================
% CONCEPT-SPECIFIC COLORS
% ===================================

% Assign consistent colors to specific concepts
% These can be used across diagrams to ensure the same concept has the same color

% Survival functions and models
\colorlet{survivalFunctionColor}{primary}
\colorlet{hazardFunctionColor}{secondary}
\colorlet{censoringColor}{quaternary}
\colorlet{kaplameierColor}{tertiary}

% Model types
\colorlet{dsmColor}{quinary}
\colorlet{mensaColor}{septenary}
\colorlet{deephitColor}{senary}

% Loss functions
\colorlet{likelihoodLossColor}{primary}
\colorlet{rankingLossColor}{secondary}
\colorlet{regressionLossColor}{tertiary}
\colorlet{classificationLossColor}{quaternary}
\colorlet{auxiliaryLossColor}{quinary}

% Events
\colorlet{event1Color}{primary}
\colorlet{event2Color}{secondary}
\colorlet{event3Color}{tertiary}
\colorlet{event4Color}{quaternary}

% Risk groups
\colorlet{lowRiskColor}{tertiary}
\colorlet{mediumRiskColor}{primary}
\colorlet{highRiskColor}{quaternary}

% ===================================
% USAGE EXAMPLES
% ===================================
%
% For tcolorbox:
% \begin{definitionbox}[title=Custom Title]
% Content
% \end{definitionbox}
%
% For TikZ:
% \begin{tikzpicture}
%   \node[concept] (a) {Concept};
%   \node[process] (b) at (2,0) {Process};
%   \draw[arrow] (a) -- (b);
% \end{tikzpicture}
%
% For PGFPlots:
% \begin{tikzpicture}
%   \begin{axis}[cycle list name=conceptcolors]
%     \addplot coordinates {(0,0) (1,1) (2,4)};
%     \addplot coordinates {(0,0) (1,2) (2,3)};
%   \end{axis}
% \end{tikzpicture}
%
% For concept-specific colors:
% \begin{tikzpicture}
%   \draw[survivalFunctionColor, thick] plot coordinates {(0,1) (1,0.8) (2,0.6)};
%   \draw[hazardFunctionColor, thick] plot coordinates {(0,0.2) (1,0.4) (2,0.6)};
% \end{tikzpicture}


% Define primary colors
\definecolor{primary}{RGB}{31, 119, 180}       % Blue
\definecolor{secondary}{RGB}{255, 127, 14}     % Orange
\definecolor{tertiary}{RGB}{44, 160, 44}       % Green
\definecolor{quaternary}{RGB}{214, 39, 40}     % Red
\definecolor{quinary}{RGB}{148, 103, 189}      % Purple
\definecolor{senary}{RGB}{140, 86, 75}         % Brown
\definecolor{septenary}{RGB}{227, 119, 194}    % Pink
\definecolor{octonary}{RGB}{127, 127, 127}     % Gray
\definecolor{nonary}{RGB}{188, 189, 34}        % Olive
\definecolor{denary}{RGB}{23, 190, 207}        % Cyan

% Define light versions for fills and backgrounds
\definecolor{primaryLight}{RGB}{174, 199, 232}
\definecolor{secondaryLight}{RGB}{255, 187, 120}
\definecolor{tertiaryLight}{RGB}{152, 223, 138}
\definecolor{quaternaryLight}{RGB}{255, 152, 150}
\definecolor{quinaryLight}{RGB}{197, 176, 213}
\definecolor{senaryLight}{RGB}{196, 156, 148}
\definecolor{septenaryLight}{RGB}{247, 182, 210}
\definecolor{octonaryLight}{RGB}{199, 199, 199}
\definecolor{nonaryLight}{RGB}{219, 219, 141}
\definecolor{denaryLight}{RGB}{158, 218, 229}

% Define dark versions for text and lines
\definecolor{primaryDark}{RGB}{17, 63, 96}
\definecolor{secondaryDark}{RGB}{179, 89, 10}
\definecolor{tertiaryDark}{RGB}{22, 80, 22}
\definecolor{quaternaryDark}{RGB}{136, 22, 22}
\definecolor{quinaryDark}{RGB}{94, 66, 121}
\definecolor{senaryDark}{RGB}{85, 52, 45}
\definecolor{septenaryDark}{RGB}{143, 75, 122}
\definecolor{octonaryDark}{RGB}{64, 64, 64}
\definecolor{nonaryDark}{RGB}{104, 104, 18}
\definecolor{denaryDark}{RGB}{14, 115, 124}

% ===================================
% TCOLORBOX STYLES
% ===================================

% Reset tcolorbox definitions with the new colors
\newtcolorbox{definitionbox}[1][]{
    enhanced,
    colback=primaryLight!60!white,
    colframe=primary!75!black,
    fonttitle=\bfseries,
    coltitle=white,
    attach boxed title to top left={yshift=-2mm, xshift=5mm},
    boxed title style={colback=primary!75!black},
    title=Definition,
    #1
}

\newtcolorbox{equationbox}[1][]{
    enhanced,
    colback=secondaryLight!60!white,
    colframe=secondary!75!black,
    fonttitle=\bfseries,
    breakable,
    coltitle=white,
    attach boxed title to top left={yshift=-2mm, xshift=5mm},
    boxed title style={colback=secondary!75!black},
    title=Equation,
    #1
}

\newtcolorbox{examplebox}[1][]{
    enhanced,
    colback=tertiaryLight!60!white,
    colframe=tertiary!75!black,
    fonttitle=\bfseries,
    coltitle=white,
    attach boxed title to top left={yshift=-2mm, xshift=5mm},
    boxed title style={colback=tertiary!75!black},
    title=Example,
    #1
}

\newtcolorbox{notebox}[1][]{
    enhanced,
    colback=quinaryLight!60!white,
    colframe=quinary!75!black,
    fonttitle=\bfseries,
    coltitle=white,
    attach boxed title to top left={yshift=-2mm, xshift=5mm},
    boxed title style={colback=quinary!75!black},
    title=Note,
    #1
}

% ===================================
% TIKZ AND PGFPLOTS STYLES
% ===================================

% TikZ styles
\tikzset{
    concept/.style={
        draw=primary,
        fill=primaryLight,
        rounded corners,
        thick,
        align=center
    },
    process/.style={
        draw=secondary,
        fill=secondaryLight,
        rectangle,
        thick,
        align=center
    },
    decision/.style={
        draw=tertiary,
        fill=tertiaryLight,
        diamond,
        thick,
        align=center
    },
    data/.style={
        draw=quaternary,
        fill=quaternaryLight,
        cylinder,
        thick,
        align=center
    },
    note/.style={
        draw=quinary,
        fill=quinaryLight,
        cloud,
        cloud puffs=15,
        thick,
        align=center
    },
    arrow/.style={
        ->,
        >=stealth,
        thick
    }
}

% PGFPlots styles
\pgfplotscreateplotcyclelist{conceptcolors}{
    {primary, mark=*, thick, mark options={fill=primary}},
    {secondary, mark=square*, thick, mark options={fill=secondary}},
    {tertiary, mark=triangle*, thick, mark options={fill=tertiary}},
    {quaternary, mark=diamond*, thick, mark options={fill=quaternary}},
    {quinary, mark=pentagon*, thick, mark options={fill=quinary}},
    {senary, mark=*, thick, mark options={fill=senary}},
    {septenary, mark=square*, thick, mark options={fill=septenary}},
    {octonary, mark=triangle*, thick, mark options={fill=octonary}},
    {nonary, mark=diamond*, thick, mark options={fill=nonary}},
    {denary, mark=pentagon*, thick, mark options={fill=denary}}
}

% ===================================
% CONCEPT-SPECIFIC COLORS
% ===================================

% Assign consistent colors to specific concepts
% These can be used across diagrams to ensure the same concept has the same color

% Survival functions and models
\colorlet{survivalFunctionColor}{primary}
\colorlet{hazardFunctionColor}{secondary}
\colorlet{censoringColor}{quaternary}
\colorlet{kaplameierColor}{tertiary}

% Model types
\colorlet{dsmColor}{quinary}
\colorlet{mensaColor}{septenary}
\colorlet{deephitColor}{senary}

% Loss functions
\colorlet{likelihoodLossColor}{primary}
\colorlet{rankingLossColor}{secondary}
\colorlet{regressionLossColor}{tertiary}
\colorlet{classificationLossColor}{quaternary}
\colorlet{auxiliaryLossColor}{quinary}

% Events
\colorlet{event1Color}{primary}
\colorlet{event2Color}{secondary}
\colorlet{event3Color}{tertiary}
\colorlet{event4Color}{quaternary}

% Risk groups
\colorlet{lowRiskColor}{tertiary}
\colorlet{mediumRiskColor}{primary}
\colorlet{highRiskColor}{quaternary}

% ===================================
% USAGE EXAMPLES
% ===================================
%
% For tcolorbox:
% \begin{definitionbox}[title=Custom Title]
% Content
% \end{definitionbox}
%
% For TikZ:
% \begin{tikzpicture}
%   \node[concept] (a) {Concept};
%   \node[process] (b) at (2,0) {Process};
%   \draw[arrow] (a) -- (b);
% \end{tikzpicture}
%
% For PGFPlots:
% \begin{tikzpicture}
%   \begin{axis}[cycle list name=conceptcolors]
%     \addplot coordinates {(0,0) (1,1) (2,4)};
%     \addplot coordinates {(0,0) (1,2) (2,3)};
%   \end{axis}
% \end{tikzpicture}
%
% For concept-specific colors:
% \begin{tikzpicture}
%   \draw[survivalFunctionColor, thick] plot coordinates {(0,1) (1,0.8) (2,0.6)};
%   \draw[hazardFunctionColor, thick] plot coordinates {(0,0.2) (1,0.4) (2,0.6)};
% \end{tikzpicture}


% Define primary colors
\definecolor{primary}{RGB}{31, 119, 180}       % Blue
\definecolor{secondary}{RGB}{255, 127, 14}     % Orange
\definecolor{tertiary}{RGB}{44, 160, 44}       % Green
\definecolor{quaternary}{RGB}{214, 39, 40}     % Red
\definecolor{quinary}{RGB}{148, 103, 189}      % Purple
\definecolor{senary}{RGB}{140, 86, 75}         % Brown
\definecolor{septenary}{RGB}{227, 119, 194}    % Pink
\definecolor{octonary}{RGB}{127, 127, 127}     % Gray
\definecolor{nonary}{RGB}{188, 189, 34}        % Olive
\definecolor{denary}{RGB}{23, 190, 207}        % Cyan

% Define light versions for fills and backgrounds
\definecolor{primaryLight}{RGB}{174, 199, 232}
\definecolor{secondaryLight}{RGB}{255, 187, 120}
\definecolor{tertiaryLight}{RGB}{152, 223, 138}
\definecolor{quaternaryLight}{RGB}{255, 152, 150}
\definecolor{quinaryLight}{RGB}{197, 176, 213}
\definecolor{senaryLight}{RGB}{196, 156, 148}
\definecolor{septenaryLight}{RGB}{247, 182, 210}
\definecolor{octonaryLight}{RGB}{199, 199, 199}
\definecolor{nonaryLight}{RGB}{219, 219, 141}
\definecolor{denaryLight}{RGB}{158, 218, 229}

% Define dark versions for text and lines
\definecolor{primaryDark}{RGB}{17, 63, 96}
\definecolor{secondaryDark}{RGB}{179, 89, 10}
\definecolor{tertiaryDark}{RGB}{22, 80, 22}
\definecolor{quaternaryDark}{RGB}{136, 22, 22}
\definecolor{quinaryDark}{RGB}{94, 66, 121}
\definecolor{senaryDark}{RGB}{85, 52, 45}
\definecolor{septenaryDark}{RGB}{143, 75, 122}
\definecolor{octonaryDark}{RGB}{64, 64, 64}
\definecolor{nonaryDark}{RGB}{104, 104, 18}
\definecolor{denaryDark}{RGB}{14, 115, 124}

% ===================================
% TCOLORBOX STYLES
% ===================================

% Reset tcolorbox definitions with the new colors
\newtcolorbox{definitionbox}[1][]{
    enhanced,
    colback=primaryLight!60!white,
    colframe=primary!75!black,
    fonttitle=\bfseries,
    coltitle=white,
    attach boxed title to top left={yshift=-2mm, xshift=5mm},
    boxed title style={colback=primary!75!black},
    title=Definition,
    #1
}

\newtcolorbox{equationbox}[1][]{
    enhanced,
    colback=secondaryLight!60!white,
    colframe=secondary!75!black,
    fonttitle=\bfseries,
    breakable,
    coltitle=white,
    attach boxed title to top left={yshift=-2mm, xshift=5mm},
    boxed title style={colback=secondary!75!black},
    title=Equation,
    #1
}

\newtcolorbox{examplebox}[1][]{
    enhanced,
    colback=tertiaryLight!60!white,
    colframe=tertiary!75!black,
    fonttitle=\bfseries,
    coltitle=white,
    attach boxed title to top left={yshift=-2mm, xshift=5mm},
    boxed title style={colback=tertiary!75!black},
    title=Example,
    #1
}

\newtcolorbox{notebox}[1][]{
    enhanced,
    colback=quinaryLight!60!white,
    colframe=quinary!75!black,
    fonttitle=\bfseries,
    coltitle=white,
    attach boxed title to top left={yshift=-2mm, xshift=5mm},
    boxed title style={colback=quinary!75!black},
    title=Note,
    #1
}

% ===================================
% TIKZ AND PGFPLOTS STYLES
% ===================================

% TikZ styles
\tikzset{
    concept/.style={
        draw=primary,
        fill=primaryLight,
        rounded corners,
        thick,
        align=center
    },
    process/.style={
        draw=secondary,
        fill=secondaryLight,
        rectangle,
        thick,
        align=center
    },
    decision/.style={
        draw=tertiary,
        fill=tertiaryLight,
        diamond,
        thick,
        align=center
    },
    data/.style={
        draw=quaternary,
        fill=quaternaryLight,
        cylinder,
        thick,
        align=center
    },
    note/.style={
        draw=quinary,
        fill=quinaryLight,
        cloud,
        cloud puffs=15,
        thick,
        align=center
    },
    arrow/.style={
        ->,
        >=stealth,
        thick
    }
}

% PGFPlots styles
\pgfplotscreateplotcyclelist{conceptcolors}{
    {primary, mark=*, thick, mark options={fill=primary}},
    {secondary, mark=square*, thick, mark options={fill=secondary}},
    {tertiary, mark=triangle*, thick, mark options={fill=tertiary}},
    {quaternary, mark=diamond*, thick, mark options={fill=quaternary}},
    {quinary, mark=pentagon*, thick, mark options={fill=quinary}},
    {senary, mark=*, thick, mark options={fill=senary}},
    {septenary, mark=square*, thick, mark options={fill=septenary}},
    {octonary, mark=triangle*, thick, mark options={fill=octonary}},
    {nonary, mark=diamond*, thick, mark options={fill=nonary}},
    {denary, mark=pentagon*, thick, mark options={fill=denary}}
}

% ===================================
% CONCEPT-SPECIFIC COLORS
% ===================================

% Assign consistent colors to specific concepts
% These can be used across diagrams to ensure the same concept has the same color

% Survival functions and models
\colorlet{survivalFunctionColor}{primary}
\colorlet{hazardFunctionColor}{secondary}
\colorlet{censoringColor}{quaternary}
\colorlet{kaplameierColor}{tertiary}

% Model types
\colorlet{dsmColor}{quinary}
\colorlet{mensaColor}{septenary}
\colorlet{deephitColor}{senary}

% Loss functions
\colorlet{likelihoodLossColor}{primary}
\colorlet{rankingLossColor}{secondary}
\colorlet{regressionLossColor}{tertiary}
\colorlet{classificationLossColor}{quaternary}
\colorlet{auxiliaryLossColor}{quinary}

% Events
\colorlet{event1Color}{primary}
\colorlet{event2Color}{secondary}
\colorlet{event3Color}{tertiary}
\colorlet{event4Color}{quaternary}

% Risk groups
\colorlet{lowRiskColor}{tertiary}
\colorlet{mediumRiskColor}{primary}
\colorlet{highRiskColor}{quaternary}

% ===================================
% USAGE EXAMPLES
% ===================================
%
% For tcolorbox:
% \begin{definitionbox}[title=Custom Title]
% Content
% \end{definitionbox}
%
% For TikZ:
% \begin{tikzpicture}
%   \node[concept] (a) {Concept};
%   \node[process] (b) at (2,0) {Process};
%   \draw[arrow] (a) -- (b);
% \end{tikzpicture}
%
% For PGFPlots:
% \begin{tikzpicture}
%   \begin{axis}[cycle list name=conceptcolors]
%     \addplot coordinates {(0,0) (1,1) (2,4)};
%     \addplot coordinates {(0,0) (1,2) (2,3)};
%   \end{axis}
% \end{tikzpicture}
%
% For concept-specific colors:
% \begin{tikzpicture}
%   \draw[survivalFunctionColor, thick] plot coordinates {(0,1) (1,0.8) (2,0.6)};
%   \draw[hazardFunctionColor, thick] plot coordinates {(0,0.2) (1,0.4) (2,0.6)};
% \end{tikzpicture}
