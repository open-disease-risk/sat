% Censoring visualization using PGFPlots with our publication-grade styles
% To be included in the book

\begin{figure}[htbp]
    \centering
    \begin{tikzpicture}
        \begin{axis}[
            publication,
            width=0.8\textwidth,
            height=0.4\textwidth,
            title={Censoring in Survival Analysis},
            xlabel={Time (months)},
            ylabel={Patient},
            ytick={1,2,3,4,5,6,7,8},
            yticklabels={Patient 1, Patient 2, Patient 3, Patient 4, Patient 5, Patient 6, Patient 7, Patient 8},
            xmin=0,
            xmax=60,
            ymin=0.25,
            ymax=8.75,
            xmajorgrids=true,
            grid style={octonary!20, very thin},
            y dir=reverse,
            legend pos=north east,
            legend style={
                font=\small,
                cells={anchor=west}
            }
        ]
            % Full observation (event occurred)
            \addplot[
                primaryDark,
                thick,
                mark=*,
                mark size=3pt,
                mark options={fill=primaryDark, solid}
            ] coordinates {
                (0,1) (32,1)
            };
            
            % Add death mark
            \node[circle, fill=quaternary, inner sep=2pt] at (axis cs:32,1) {};
            
            % Right-censored observation
            \addplot[
                primaryDark,
                thick,
                mark=none
            ] coordinates {
                (0,2) (48,2)
            };
            
            % Add censoring mark
            \draw[censoringColor, very thick] 
                (axis cs:48-1.5,2-1.5) -- (axis cs:48+1.5,2+1.5);
            \draw[censoringColor, very thick] 
                (axis cs:48-1.5,2+1.5) -- (axis cs:48+1.5,2-1.5);
                
            % Late entry (left truncation)
            \addplot[
                primaryDark,
                thick,
                mark=none,
                dash pattern=on 2pt off 2pt
            ] coordinates {
                (0,3) (12,3)
            };
            
            \addplot[
                primaryDark,
                thick,
                mark=none
            ] coordinates {
                (12,3) (40,3)
            };
            
            % Add death mark
            \node[circle, fill=quaternary, inner sep=2pt] at (axis cs:40,3) {};
            
            % Left truncation and right censoring
            \addplot[
                primaryDark,
                thick,
                mark=none,
                dash pattern=on 2pt off 2pt
            ] coordinates {
                (0,4) (15,4)
            };
            
            \addplot[
                primaryDark,
                thick,
                mark=none
            ] coordinates {
                (15,4) (52,4)
            };
            
            % Add censoring mark
            \draw[censoringColor, very thick] 
                (axis cs:52-1.5,4-1.5) -- (axis cs:52+1.5,4+1.5);
            \draw[censoringColor, very thick] 
                (axis cs:52-1.5,4+1.5) -- (axis cs:52+1.5,4-1.5);
                
            % Interval censoring
            \addplot[
                primaryDark,
                thick,
                mark=none
            ] coordinates {
                (0,5) (25,5)
            };
            
            \addplot[
                primaryDark,
                thick,
                mark=none,
                dash pattern=on 3pt off 3pt
            ] coordinates {
                (25,5) (38,5)
            };
            
            \addplot[
                primaryDark,
                thick,
                mark=none
            ] coordinates {
                (38,5) (45,5)
            };
            
            % Add death mark
            \node[circle, fill=quaternary, inner sep=2pt] at (axis cs:45,5) {};
            
            % Add interval censoring annotation arrow - moved up to avoid overlap with Event type 1
            \draw[->, thin, gray] (32,6.7) -- (32,5.2);
            \node[above, align=center, font=\footnotesize, text=gray] at (32,6.8) {Interval\\censoring};
            
            % Event observed in competing risks
            \addplot[
                primaryDark,
                thick,
                mark=none
            ] coordinates {
                (0,6) (28,6)
            };
            
            % Event 1 mark
            \node[circle, fill=event1Color, inner sep=2pt] at (axis cs:28,6) {};
            
            % Another competing risk event
            \addplot[
                primaryDark,
                thick,
                mark=none
            ] coordinates {
                (0,7) (36,7)
            };
            
            % Event 2 mark
            \node[circle, fill=event2Color, inner sep=2pt] at (axis cs:36,7) {};
            
            % Another competing risk with censoring
            \addplot[
                primaryDark,
                thick,
                mark=none
            ] coordinates {
                (0,8) (50,8)
            };
            
            % Censoring mark
            \draw[censoringColor, very thick] 
                (axis cs:50-1.5,8-1.5) -- (axis cs:50+1.5,8+1.5);
            \draw[censoringColor, very thick] 
                (axis cs:50-1.5,8+1.5) -- (axis cs:50+1.5,8-1.5);
            
            % Add labels with arrows to explain censoring concepts clearly
            % Complete observation
            \draw[->, thin, gray] (33,1.3) -- (33,1.1);
            \node[above, align=center, font=\footnotesize, text=gray] at (33,1.3) {Complete\\observation};
            
            % Right censoring
            \draw[->, thin, gray] (49,2.3) -- (49,2.1);
            \node[above, align=center, font=\footnotesize, text=gray] at (49,2.3) {Right\\censoring};
            
            % Left truncation (Patient 3)
            \draw[->, thin, gray] (6,3.4) -- (6,3.1);
            \node[above, align=center, font=\footnotesize, text=gray] at (6,3.4) {Left\\truncation};
            
            % Event mark
            \draw[->, thin, gray] (41,3.3) -- (41,3.1);
            \node[above, align=center, font=\footnotesize, text=gray] at (41,3.3) {Event};
            
            % Left truncation (Patient 4)
            \draw[->, thin, gray] (10,4.4) -- (14,4.1);
            \node[above, align=center, font=\footnotesize, text=gray] at (10,4.4) {Left\\truncation};
            
            % Right censoring (Patient 4)
            \draw[->, thin, gray] (53,4.3) -- (53,4.1);
            \node[above, align=center, font=\footnotesize, text=gray] at (53,4.3) {Right\\censoring};
            
            % Event after interval
            \draw[->, thin, gray] (46,5.3) -- (46,5.1);
            \node[above, align=center, font=\footnotesize, text=gray] at (46,5.3) {Event};
            
            % Event type 1 - shifted to the right where there's more space
            \draw[->, thin, gray] (40,6.3) -- (29,6.1);
            \node[above, align=center, font=\footnotesize, text=gray] at (40,6.3) {Event\\type 1};
            
            % Event type 2 - shifted to the right where there's more space
            \draw[->, thin, gray] (45,7.3) -- (37,7.1);
            \node[above, align=center, font=\footnotesize, text=gray] at (45,7.3) {Event\\type 2};
            
            % Censored
            \draw[->, thin, gray] (51,8.3) -- (51,8.1);
            \node[above, align=center, font=\footnotesize, text=gray] at (51,8.3) {Censored};
            
            % Add a legend at the bottom of the plot instead of top-right
            \legend{}
            
            % Create a separate legend below the plot using nodes
            \coordinate (legendpos) at (rel axis cs:0.5,-0.25);
            
            % Symbol for event of interest
            \node[inner sep=0] at ($(legendpos)+(-4.5,0)$) {\tikz{\draw[mark size=3pt, mark=*, mark options={fill=quaternary}] plot coordinates {(0,0)};}};
            \node[right, font=\footnotesize] at ($(legendpos)+(-4.3,0)$) {Event of interest};
            
            % Symbol for competing event type 1
            \node[inner sep=0] at ($(legendpos)+(-1.5,0)$) {\tikz{\draw[mark size=3pt, mark=*, mark options={fill=event1Color}] plot coordinates {(0,0)};}};
            \node[right, font=\footnotesize] at ($(legendpos)+(-1.3,0)$) {Competing event type 1};
            
            % Symbol for competing event type 2
            \node[inner sep=0] at ($(legendpos)+(1.5,0)$) {\tikz{\draw[mark size=3pt, mark=*, mark options={fill=event2Color}] plot coordinates {(0,0)};}};
            \node[right, font=\footnotesize] at ($(legendpos)+(1.7,0)$) {Competing event type 2};
            
            % Symbol for censoring
            \node[inner sep=0] at ($(legendpos)+(-4.5,-0.5)$) {\tikz{\draw[censoringColor, very thick] (-0.1,-0.1) -- (0.1,0.1); \draw[censoringColor, very thick] (-0.1,0.1) -- (0.1,-0.1);}};
            \node[right, font=\footnotesize] at ($(legendpos)+(-4.3,-0.5)$) {Censoring};
            
            % Symbol for observation time
            \node[inner sep=0] at ($(legendpos)+(-1.5,-0.5)$) {\tikz{\draw[primaryDark, thick] (-0.15,0) -- (0.15,0);}};
            \node[right, font=\footnotesize] at ($(legendpos)+(-1.3,-0.5)$) {Observation time};
            
            % Symbol for unobserved time
            \node[inner sep=0] at ($(legendpos)+(1.5,-0.5)$) {\tikz{\draw[primaryDark, thick, dash pattern=on 2pt off 2pt] (-0.15,0) -- (0.15,0);}};
            \node[right, font=\footnotesize] at ($(legendpos)+(1.7,-0.5)$) {Unobserved time};
        \end{axis}
    \end{tikzpicture}
    \caption{Different types of censoring in survival analysis and competing risks. The plot shows various scenarios including complete observation, right censoring, left truncation, interval censoring, and competing risks. Solid lines represent observed follow-up, dashed lines represent unobserved time, and markers indicate events or censoring.}
    \label{fig:censoring}
\end{figure}

% Alternative version with focus on competing risks
\begin{figure}[htbp]
    \centering
    \begin{tikzpicture}
        \begin{axis}[
            publication,
            width=0.8\textwidth,
            height=0.35\textwidth,
            title={Competing Risks Visualization},
            xlabel={Time (months)},
            ylabel={},
            xmin=0,
            xmax=100,
            ymin=0,
            ymax=1,
            ymajorgrids=true,
            grid style={octonary!20, very thin},
            legend pos=north east,
            legend style={
                font=\small,
                cells={anchor=west}
            }
        ]
            % Overall Survival
            \addplot[
                survivalFunctionColor,
                custom ultra thick,
            ] coordinates {
                (0,1) (10,0.95) (20,0.9) (30,0.82) (40,0.7)
                (50,0.58) (60,0.45) (70,0.35) (80,0.27) (90,0.2) (100,0.15)
            };
            
            % Cause-specific survival curves
            \addplot[
                event1Color,
                thick,
            ] coordinates {
                (0,1) (10,0.98) (20,0.95) (30,0.9) (40,0.85)
                (50,0.78) (60,0.72) (70,0.68) (80,0.65) (90,0.62) (100,0.6)
            };
            
            \addplot[
                event2Color,
                thick,
            ] coordinates {
                (0,1) (10,0.97) (20,0.94) (30,0.9) (40,0.82)
                (50,0.73) (60,0.65) (70,0.58) (80,0.53) (90,0.48) (100,0.42)
            };
            
            \addplot[
                event3Color,
                thick,
            ] coordinates {
                (0,1) (10,0.99) (20,0.97) (30,0.92) (40,0.82)
                (50,0.74) (60,0.64) (70,0.52) (80,0.43) (90,0.35) (100,0.30)
            };
            
            % Add annotations with arrows for better readability
            \draw[->, thin] (60,0.3) -- (65,0.35);
            \node[left, align=right, font=\footnotesize] at (60,0.3) {Overall\\survival};
            
            \draw[->, thin, event1Color] (60,0.7) -- (65,0.68);
            \node[left, align=right, font=\footnotesize, text=event1Color] at (60,0.7) {Event 1-specific\\survival};
            
            \draw[->, thin, event2Color] (85,0.58) -- (80,0.58);
            \node[right, align=left, font=\footnotesize, text=event2Color] at (85,0.58) {Event 2-specific\\survival};
            
            \draw[->, thin, event3Color] (85,0.48) -- (80,0.48);
            \node[right, align=left, font=\footnotesize, text=event3Color] at (85,0.48) {Event 3-specific\\survival};
            
            % Cumulative incidence curves
            \addplot[
                event1Color,
                dashed,
                thick,
            ] coordinates {
                (0,0) (10,0.02) (20,0.05) (30,0.10) (40,0.15)
                (50,0.22) (60,0.28) (70,0.32) (80,0.35) (90,0.38) (100,0.40)
            };
            
            \addplot[
                event2Color,
                dashed,
                thick,
            ] coordinates {
                (0,0) (10,0.03) (20,0.06) (30,0.1) (40,0.18)
                (50,0.27) (60,0.35) (70,0.42) (80,0.47) (90,0.52) (100,0.58)
            };
            
            \addplot[
                event3Color,
                dashed,
                thick,
            ] coordinates {
                (0,0) (10,0.01) (20,0.03) (30,0.08) (40,0.18)
                (50,0.26) (60,0.36) (70,0.48) (80,0.57) (90,0.65) (100,0.70)
            };
            
            % Add annotations for CIF with improved positioning and arrows
            \draw[->, thin, event1Color] (15,0.07) -- (25,0.09);
            \node[left, align=right, font=\footnotesize, text=event1Color] at (15,0.07) {CIF Event 1};
            
            \draw[->, thin, event2Color] (25,0.15) -- (35,0.17);
            \node[left, align=right, font=\footnotesize, text=event2Color] at (25,0.15) {CIF Event 2};
            
            \draw[->, thin, event3Color] (35,0.22) -- (45,0.25);
            \node[left, align=right, font=\footnotesize, text=event3Color] at (35,0.22) {CIF Event 3};
            
            % Stacked CIF curve showing all events
            \addplot[
                octonaryDark,
                dotted,
                thick,
            ] coordinates {
                (0,0) (10,0.06) (20,0.14) (30,0.28) (40,0.51)
                (50,0.75) (60,0.99) (70,1.22) (80,1.39) (90,1.55) (100,1.68)
            };
            
            % Legend
            \addlegendentry{Overall survival};
            \addlegendentry{Event 1-specific survival};
            \addlegendentry{Event 2-specific survival};
            \addlegendentry{Event 3-specific survival};
            \addlegendentry{CIF for Event 1};
            \addlegendentry{CIF for Event 2};
            \addlegendentry{CIF for Event 3};
            \addlegendentry{Sum of CIFs};
        \end{axis}
    \end{tikzpicture}
    \caption{Visualization of competing risks: survival curves and cumulative incidence functions (CIFs). The overall survival (thick blue line) decreases more rapidly than any cause-specific survival curve. The dashed lines show the cumulative incidence functions for each competing event. Note that the sum of all CIFs (dotted line) can exceed 1.0 when viewing causes independently, illustrating why proper competing risks analysis is important.}
    \label{fig:competing-risks}
\end{figure}